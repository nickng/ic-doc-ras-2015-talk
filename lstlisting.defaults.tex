%
% Default settings for lstlisting
%

\newcommand{\lstCodeSize}{\footnotesize}
\newcommand{\lstPrimitiveStyle}{\color{Blue}\bfseries}
\newcommand{\lstDatatypeStyle}{\color{OliveGreen}\bfseries}
\newcommand{\lstConstStyle}{\color{RedViolet}\bfseries}
\newcommand{\lstAuxStyle}{\color{RedViolet}}
\newcommand{\lstNumberStyle}{\tiny\sffamily\color{Gray}}
\newcommand{\lstInlineStyle}{\footnotesize\ttfamily}

\lstset{
  float=hbp,
  backgroundcolor=\color{gray-bg},
  basicstyle=\lstCodeSize\ttfamily,
  identifierstyle=\color{Black},
  stringstyle=\color{Violet},
  commentstyle=\itshape\color{Gray},
  keywordstyle=\lstPrimitiveStyle,
  columns=flexible,
  tabsize=2,
  extendedchars=true,
  captionpos=b,
  numberstyle=\lstNumberStyle,
  numbers=left,
  xleftmargin=2em, % Don't go over the borders
  framexleftmargin=1.5em,
  showstringspaces=false,
  escapeinside={/*<}{>*/},
  mathescape=true,
  breaklines=true,
}
\lstdefinestyle{intable}{
  backgroundcolor=,
  numbers=none,
  breaklines=true,
  escapeinside={/*<}{>*/},
  columns=flexible
}
\lstdefinestyle{process}{
  frame=l,
  backgroundcolor=\color{gray-bg},
  numbers=none,
  breaklines=true,
  escapeinside={/*<}{>*/},
  columns=flexible,
  xleftmargin=0pt,
  framexleftmargin=0pt,
  linewidth=0.8\textwidth
}
